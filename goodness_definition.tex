\subsection{Definition}

Consider a horizontal rectangle $\D = \left\{ a_{i_1} ... a_{i_2} \right\}$.

\begin{definition}
	Rectangle $\D$ is called \textbf{good},
	if subrectangles
	$\{2, 0\}$ and $\{1, 0\}$ intersect.
\end{definition}

\pic[0.5][Good rectangles if $i_1$ is odd (upper) and even (lower).]{goodness}

For example, in case $i_1$ is even, goodness is equivalent to

\begin{equation}\label{goodness_i1_even}
	\{2, 0\}'' \geqslant \{1, 0\}'.
\end{equation}

Bounds of rectangles are determined by the rules from section \oldref{sc:boundaries}.

If the rectangle $\D$ is not good (for example, \ref{goodness_i1_even} doesn't take place), then
$$ ( \{2, 0\}'' ; \{1, 0\}' ) \not\subset \D. $$

Clearly, if the rectangle is not good,
then one can not split it into smaller subrectangles whose projections cover the projection of initial one.
That's why we will only consider good rectangles during the proof.

%This is why goodness is necessary for the way we want to prove Freiman's constant.
