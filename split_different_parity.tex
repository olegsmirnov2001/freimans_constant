\section{Induction, $i_1 \not\equiv i_2 \pmod 2$}

Let $\D = \{ a_{i_1}, ..., a_{i_2}\}$ be a horizontal rectangle,
satisfying conditions from sections \oldref{sc:boundaries} and \oldref{sc:good}.

This section deals with the case $i_1 \not\equiv i_2 \pmod 2$.

It is easier, because in this case both the lower-left and upper-right corners
are filled with the ``wide'' rectangles $\{1,\}$ and $\{,1\}$ (see figure \oldref{fg:subrectangles-10-01}).
In case $i_1 \equiv i_2 \pmod 2$ the lower-left corner, filled with $\{3,3\}$,
will be a pain in the neck.

Without loss of generality, we may assume that $i_1$ is even and, therefore, $i_2$ is odd.
As the reader will see, the agrument is repeated for the opposite parity,
up to differences in illustrations.

\subsection{Case $\{, 1\}$ is good}

At first, suppose that the subrectangle $\{,1\}$ is good.

It is so, if

\pic[0.5]{induction-odd-2}

\begin{equation}\label{14.1}
	\g - \g' < \d - \d',
\end{equation}
where
\Tp{66}{63}{70}{90}
which can be rewritten with \ref{eq:aspect-ratio} as follows:
\begin{equation}\label{14.2}
	0,74 < q \up{63}{66} \up{70}{90}.
\end{equation}

Constants $\T{\g}$ and $\T{\d}$ are chosen to ensure that
\begin{equation*}
	\left[ \d, \d' \right] \subset \D_2,
\end{equation*}
regardless of the rules from the section \oldref{sc:boundaries} defining $\D_2'$ and $\D_2''$.

From \ref{14.2} it follows that the original rectangle $\D$ is contained
within the union of the subrectangles $\{1,\}$ and $\{,1\}$.

\pic[0.4][\label{fg:subrectangles-10-01} Subrectangles $\{1,\}$ and $\{,1\}$.]{induction-odd-1}

Now we will try contradiction and suppose that
\begin{equation}\label{14.3}
	0,74 \geqslant q \up{63}{66} \up{70}{90}.
\end{equation}

We will use the inequality \ref{14.3} until the end of the section.


\subsection{Case $\D$ is left-shortened}

Now suppose that the following inequality takes place:
\pic[0.5]{induction-odd-3}
\begin{equation*}
	\g - \g' > \d - \d',
\end{equation*}
where
\Tp{63}{35}{63}{70}
which can be rewritten as
\begin{equation}\label{14.5}
	0,558 > q \up{35}{63} \up{63}{70}.
\end{equation}

If the rectangle $\D$ is of type $S-S$ or $S-N$,
that is, the lower bound $\D'$ is defined using rules \ref{eq:bound-l-odd-sn} or \ref{eq:bound-l-odd-ss},
then the splitting is finished, as $\D$ is contained within the union of $\{1,\}$ and $\{2,\}$.


\subsection{Case $\D$ is left-normal}

Now turn to case if \ref{14.5} takes place,
and $\D$ is of type $N-S$ or $N-N$,
so $\D'$ is set with one of the rules \ref{eq:bound-l-odd-nn} or \ref{eq:bound-l-odd-ns}.

%Obtain the condition, equivalent to ``some other condition -- to be clarified''.
Let's see when the subrectangle $\{3, 11\}$ is right-normal.
The following condition should take place:

\pic[0.5]{induction-odd-4}

\begin{equation*}
	\g - \g' > 1,4 (\d - \d'),
\end{equation*}
where
\Tp{28}{22}{68}{65}
which can be written as
\begin{equation}\label{14.6}
	0,345 > q \up{22}{65} \up{28}{68}.
\end{equation}

\pic[0.5]{pic5}
If the condition \ref{14.6} takes place, then
the inequality
\begin{equation*}
	\g - \g' > \d - \d',
\end{equation*}
where
\Tp{25}{3}{66}{63}
which can be written as
\begin{equation}\label{14.7}
	0,322 > q \up{3}{63} \up{25}{66}.
\end{equation}

All in all, if rectangle $\D$ satisfies both conditions \ref{14.6} and \ref{14.7},
then it can be split into $\{1,\}$, $\{2,\}$ and $\{3,\}$.

Now suppose that at least one of \ref{14.6} and \ref{14.7} doesn't take place.
We will show that then we can split $\D$ into $\{1,\}$, $\{2,\}$ and $\{3,1\}$.

First, we need to check goodness of subcovering $\{3,1\}$.
We will prove that it meets the sufficient condition \ref{eq:goodness-condition-3,43}.
We need to show that
\begin{equation}\label{14.8}
	3,43 (\g - \g') > \left| \d - \d' \right|,
\end{equation}
where
\Tp[.]{28}{1}{95}{65}

Inequality \ref{14.8} is equivalent to
\begin{equation*}
	0,628 > q \up{1}{65} \up{28}{95}.
\end{equation*}

Recalling \ref{14.5}, we have to show that
\begin{equation*}
	0,558 \pu{63}{35} \pu{70}{63}
	< 0,627 \pu{65}{1} \pu{95}{28},
\end{equation*}
and it is true.

Now show that the subrectangles $\{3,1\}$ and $\{2, 1\}$ intersect.

First, let $a_{i_2} = 3$, so $\D_2'$ is set with the formula \ref{eq:bound-l-odd-ns}.

The following inequality should take place:
\begin{equation}\label{14.9}
	\g - \g' < \d - \d',
\end{equation}
where
\Tp{30}{25}{70}{90}
which is transformed into
\begin{equation}\label{14.10}
	0,328 < q \up{25}{70} \up{30}{90}.
\end{equation}

If \ref{14.6} doesn't take place, then we need the following inequality for \ref{14.10}:
\begin{gather*}
	0,345 \pu{65}{22} \pu{68}{28} > 0,328 \pu{70}{25} \pu {90}{30}\\
	\ArrowBetweenLines
	1,05 > \pp {70}{65} \pp{90}{68} \uu{22}{25} \uu{28}{30}.
\end{gather*}

From $a_{i_2} = 3$ it follows that $p' \leqslant \frac{1}{3}$.
Setting $p' = \frac{1}{3}$ and $p = \frac{1}{4}$ in the rhs, increasing it and obtaining the correct inequality.
Thus, it is correct.

Now suppose that \ref{14.6} takes place, while \ref{14.7} doesn't.
Setting $\T{\d} = \T{66}$ (instead of $\T{70}$) in \ref{14.9}, get the following inequality:

\begin{equation}\label{14.11}
	0,288 < q \up{25}{66} \up{30}{90}.
\end{equation}

Inequality \ref{14.11} follows from denial of inequality \ref{14.7}, if
\begin{gather*}
	0,322 \pu{63}{3} \pu{66}{25} > 0,298 \pu{66}{25} \pu{90}{30}\\
	\ArrowBetweenLines
	1,08 > \uu{3}{30} \pp{90}{63}.
\end{gather*}

Setting $p = \frac{1}{4}$ and $p' = \frac{1}{3}$, easily check the last inequality.

\subsection{Case we can consider $\{2, 1\}$}

Now consider the case, when $a_{i_2} \ne 3$ and $\D_1'$ is defined with \ref{eq:bound-l-odd-nn}.

For the subrectangle $\{2, 1\}$ left-normality we need
\begin{equation}\label{14.12}
	1,4 (\g - \g') < \| \d - \d' \|,
\end{equation}
where
\Tp[,]{33}{29}{95}{92}
so \ref{14.12} can be rewritten as
\begin{equation}\label{14.13}
	0,182 < q \up{29}{92} \up{33}{95}.
\end{equation}

If neither \ref{14.6} nor \ref{14.7} takes place, then it's easy to check \ref{14.13}.

In \ref{14.9} we can set $\T{\g'} = \T{94}$ (instead of $\T{90}$), so \ref{14.9} is transformed into
\begin{equation}\label{14.14}
	0,248 < q \up{25}{70} \up{30}{94}.
\end{equation}

If \ref{14.6} doesn't take place, then for \ref{14.14} we will check the following:

\begin{gather*}
	0,345 \pu{65}{22} \pu{68}{28} > 0,248 \pu{70}{25} \pu{94}{30}\\
	\ArrowBetweenLines
	1,39 > \pp{70}{65} \pp{94}{68} \uu{22}{25} \uu{28}{30},
\end{gather*}
which is clear.

If \ref{14.7} doesn't take place, then for \ref{14.14} we will check
\begin{gather*}
	0,322 \pu{63}{3} \pu{66}{25} > 0,248 \pu{70}{25} \pu{94}{30}\\
	\ArrowBetweenLines
	1,29 > \uu{3}{30} \pp{70}{63} \pp{94}{66},
\end{gather*}
which is true.

\subsection{Further plan}

It remains for us to check the case if \ref{14.3} takes place and \ref{14.5} doesn't --
instead, having the negation of \ref{14.5}:
\begin{equation}\label{14.15}
	0,558 \leqslant q \up{35}{63} \up{63}{70}.
\end{equation}

We have proved that we can consider subrectangles $\{1,\}$ and $\{2, 2\}$.
\pic[0.5]{max-2-2}
The picture illustrates the case when
the boundaries $\D_1'$ and $\D_2''$ of segments $\D_1$ and $\D_2$ are defined
with constants $\T{\D_1'} = \T{\D_2''} = \T{30} = \left[0; 213\overline{12}\right]$.
In this case, having, in particular, $a_{i_1} = 1$, $a_{i_1 + 1} = 3$.
We set segment $\D_1$ to be left-shortened to forbid $a_{i_1 - 1} = 3$,
because it can happen that we will face the forbidden combination $(31313)$.

So, in the illustrated case, the only unused subsegment is $\{2, 1\}$.
However, subsegments $\{2, 1\}$ and $\{2, 2\}$ doesn't intersect,
so we will be unable to construct a family of subsegments, covering $\D$.

How to overcome this difficulty?
The idea being developed is to use $a_{i_1 - 1} = 3$ (and $a_{i_2 + 1} = 3$),
but set $a_{i_1 - 2},\, a_{i_1 - 3}$ (and $a_{i_2 + 2},\, a_{i_2 + 3}$),
which produce the thinner subrectangles and doesn't occur in string $(31313)$.

\subsection{List of subrectangles}

Let's turn to formal plan.

If $\D_1'$ and $\D_2''$ were set with $\T{3} = \left[0; 3\overline{12}\right]$,
then we would have been able to take $\{3, 3\}$, which intersects with $\{2, 2\}$.

Instead of $\{3, 3\}$, we will take the following list of subrectangles:
\begin{gather*}
	\{23, 32\}, \{23, 33\} \,(\text{or } \{223, 33\} \text{ and } \{323, 33\}), \{113, 311\},\\
	\{213, 312\}, \{213, 311\}, \{113, 32\}, \{113, 33\} \,(\text{or } \{1113, 33\}), \{213, 33\}.
\end{gather*}
\pic[0.9][
	Illustration of the subrectangles replacing the subrectangle $\{3,3\}$.\\
	I will make a 2d illustration in the future.
]{subrectangles-instead-3-3}

The next subrectangle to consider is $\{3,2\}$, but instead we will consider the following subrectangles:
\begin{equation*}
	\{33, 213\}, \{33, 212\}, \{23, 213\}, \{23, 212\}, \{23, 211\}.
\end{equation*}

\subsection{Correctness proof}

We will show that the projections of adjacent subrectangles intersect and every of them is good.

Show that $\{23,32\}$ is good.
Indeed, the condition
\begin{equation*}
	\g - \g' < 3,43(\d - \d'),
\end{equation*}
where
\Tp[,]{21}{15}{15}{21}
is equivalent to
\begin{equation*}
	0,3 < q \up{15}{15} \up{21}{21}
\end{equation*}
and follows from \ref{14.15}.

Subrectangle $\{23, 32\}$ intersects with $\{2,2\}$. The following inequality is required:
\begin{equation*}
	\g - \g' < \d - \d',
\end{equation*}
where
\Tp[,]{30}{19}{16}{59}
so obtain condition
\begin{equation*}
	0,4612 < q \up{19}{16} \up{30}{59},
\end{equation*}
which follows from 15:
\begin{gather*}
	0,558 \pu{63}{35} \pu{70}{63} > 0,4612 \pu{16}{19} \pu{59}{30}\\
	\ArrowBetweenLines
	1,2 > \pp{16}{63} \pp{59}{70} \uu{35}{19} \uu{63}{30},
\end{gather*}
and the last line is obvious.

Checking the goodness of $\{23, 33\}$.
The condition
\begin{equation*}
	\g - \g' < 3,43(\d - \d')
\end{equation*}
should take place for
\Tp[,]{21}{15}{22}{28},
and it can be rewritten as follows:
\begin{equation}\label{14.16}
	0,533 < q \up{15}{22} \up{21}{28}.
\end{equation}

If \ref{14.16} takes place, then we take $\{23, 33\}$.
Checking that $\{23, 33\}$ intersects with $\{23, 32\}$ is easy.
If \ref{14.16} doesn't take place, then we take $\{223, 33\}$ and $\{323, 33\}$ instead.

At first, showing that $\{223, 33\}$ and $\{23, 32\}$ intersect.

Subrectangle $\{223, 33\}$ is right-normal, since
\begin{equation*}
	1,4 (\g - \g' < \d - \d'),
\end{equation*}
where
\begin{equation*}
	\begin{aligned}
		&\T{\g} = \left[0; 322 \overline{31}\right] = 0,293294,\\
		&\T{\g'} = \left[0; 3223\overline{31}\right] = 0,293176,\\
		&\T{\d} = \T{22},\\
		&\T{\d'} = \left[0;333\overline{31}\right] = 0,302806,
	\end{aligned}
\end{equation*}
transforms into
\begin{equation*}
	0,457 < q \up{\g}{\d} \up{\g'}{'d'},
\end{equation*}
and it follows from \ref{14.15}, if
\begin{equation*}
	1,22 > \pp{\d}{63} \pp{\d'}{70} \uu{35}{\g} \uu{63}{\g'},
\end{equation*}
which is clear.

Now the condition
\begin{equation*}
	\g - \g' > \d - \d',
\end{equation*}
where
\begin{equation*}
	\T{\g} = \left[0; 3223\overline{12}\right] = 0,293283,\; \T{\g'} = \T{16},\; \T{\d} = \T{20},\; \T{\d'} = \T{23},
\end{equation*}
transforms into
\begin{equation*}
	0,561 > q \up{16}{20} \up{\g}{23},
\end{equation*}
and it follows from negation of \ref{14.16}.

Let's try to take next subrectangle $\{113, 32\}$.
Let's see if it intersects with $\{23, 33\}$ (or $\{323, 33\}$).
The condition of intersection
\begin{equation*}
	\g - \g' < \d - \d',
\end{equation*}
where
\Tp[,]{16}{13}{18}{25}
is equivalent to
\begin{equation}\label{14.17}
	0,7 < q \up{13}{18} \up{16}{25}.
\end{equation}

Assume that \ref{14.17} doesn't take place:
\begin{equation}\label{14.18}
	0,7 \geqslant a \up{13}{18} \up{16}{25}.
\end{equation}

Take subrectangle $\{113, 311\}$ and show that it intersects with $\{23, 33\}$.
The following inequality should take place:
\begin{equation*}
	\g - \g' < \d - \d',
\end{equation*}
where
\Tp{16}{13}{10}{25}

Obtain
\begin{equation*}
	0,314 < q \up{13}{10} \up{16}{25},
\end{equation*}
which follows from \ref{14.15}.

Now show that $\{213, 312\}$ intersects with $\{113, 311\}$.

The condition
\begin{equation*}
	\g - \g' < \d - \d',
\end{equation*}
where
\Tp[,]{8}{5}{3}{11},
which transforms into
\begin{equation*}
	0,338 < q \up{3}{3} \up{8}{11},
\end{equation*}
and it follows from \ref{14.15}.

Show that $\{213, 311\}$ intersects with $\{213, 312\}$.
The condition
\begin{equation*}
	\g - \g' > \d - \d',
\end{equation*}
where
\Tp[,]{5}{3}{5}{8}
transforms into
\begin{equation}\label{14.19}
	0,77 > q \up{3}{5} \up{5}{8}.
\end{equation}
Inequality \ref{14.19} follows from \ref{14.18}, if
\begin{gather*}
	0,7 \pu{18}{13} \pu{25}{16} < 0,77 \pu{5}{3} \pu{8}{5}\\
	\ArrowBetweenLines
	\pp{18}{5} \pp{25}{8} \uu{3}{13} \uu{5}{16} < 1,1,
\end{gather*}
which is true.

Constant $\T{\d'}$ can be set to $\T{8}$, only if $\{213, 311\}$ is right-normal, for which the condition
\begin{equation*}
	\g - \g' > 1,4 (\d - \d'),
\end{equation*}
where
\Tp[,]{9}{7}{4}{6}
should take place.
It is equivalent to
\begin{equation*}
	0,8 > q \up{7}{4} \up{9}{6},
\end{equation*}
which is true in consideration of \ref{14.18}.

Finally, it is very easy to check that $\{113, 32\}$ intersects with $\{213, 311\}$.

Overall, regardless of whether condition 17 is satisfied or not, we come to subrectangle $\{113, 32\}$.
Further we note that it intersects with $\{113, 33\}$.
We will check here that the latter one is good.
The following inequality is required:
\begin{equation*}
	\g - \g' < 3,8 (\d - \d'),
\end{equation*}
where
\Tp[,]{14}{7}{22}{28}
and obtain
\begin{equation*}
	0,44 < q \up{7}{22} \up{14}{28},
\end{equation*}
which follows from \ref{14.15}.

It is easily checked that $\{113, 33\}$ intersects with $\{213, 32\}$.

Let's find the condition for the subrectangle $\{2, 11\}$ to be right-normal.
\begin{equation*}
	\g - \g' > 1,4 (\d - \d'),
\end{equation*}
where
\Tp[,]{64}{61}{65}{68}
which is equivalent to
\begin{equation}\label{14.20}
	0,714 > q \up{61}{65} \up{64}{68}.
\end{equation}

Let's find the condition for subrectangle $\{2, 11\}$ to intersect with $\{213, 32\}$:
\begin{equation*}
	\g - \g' > \d - \d',
\end{equation*}
where
\Tp[,]{63}{3}{20}{66}
which is equivalent to
\begin{equation}\label{14.21}
	0,662 > q \up{3}{20} \up{63}{66}.
\end{equation}

The condition \ref{14.20} follows from \ref{14.21}, because
\begin{gather*}
	0,662 > q \pu{30}{3} \pu{66}{63} < 0,714 \pu{65}{61} \pu{68}{64}\\
	\ArrowBetweenLines
	\pp{20}{65} \pp{66}{68} \uu{61}{3} \uu{64}{63} < 1,078,
\end{gather*}
which is easily checked.

In consideration of \ref{14.21}, we come to $\{2,11\}$.

Now suppose that \ref{14.22}, the negation of \ref{14.21}, takes place:
\begin{equation}\label{14.22}
	0,662 \leqslant q \up{3}{20} \up{63}{66}.
\end{equation}

Return to condition \ref{14.1}.
