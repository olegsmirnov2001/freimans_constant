\section{Induction, $i_1 \not\equiv i_2 \pmod 2$}

Let $\D = \{ a_{i_1}, ..., a_{i_2}\}$ be a horizontal rectangle,
satisfying conditions from sections \oldref{sc:boundaries} and \oldref{sc:good}.

This section deals with the case $i_1 \not\equiv i_2 \pmod 2$.

Without loss of generality, we may assume that $i_1$ is even and, therefore, $i_2$ is odd.
As the reader will see, the agrument is repeated for the different parity,
up to differences in illustrations.

\subsection{Case $\{, 1\}$ is good}

At first, suppose that the subrectangle $\{,1\}$ is good.

It is so, if
\begin{equation*}
	\g - \g' < \d - \d',
\end{equation*}
where
\Tp{66}{63}{70}{90}
which can be rewritten with \ref{eq:aspect-ratio} as follows:
\begin{equation}\label{14.2}
	0,74 < q \up{63}{66} \up{70}{90}.
\end{equation}

Constants $\T{\g}$ and $\T{\d}$ are chosen to ensure that
\begin{equation*}
	\left[ \d, \d' \right] \subset \D_2,
\end{equation*}
regardless of the rules from the section \oldref{sc:boundaries} defining $\D_2'$ and $\D_2''$.

From \ref{14.2} it follows that the original rectangle $\D$ is contained
within the union of the subrectangles $\{1,\}$ and $\{,1\}$.

Now we will try contradiction and suppose that
\begin{equation}\label{14.3}
	0,74 \geqslant q \up{63}{66} \up{70}{90}.
\end{equation}

We will use the inequality \ref{14.3} until the end of the section.


\subsection{Case resection $\left[ \D_1(1)''; \D_1(2)' \right] \in \D$ is normal}

Now suppose that the following inequality takes place:
\begin{equation*}
	\g - \g' > \d - \d',
\end{equation*}
where
\Tp{63}{35}{63}{70}
which can be rewritten as
\begin{equation}\label{14.5}
	0,558 > q \up{35}{63} \up{63}{70}.
\end{equation}

If the rectangle $\D$ is of type $S-S$ or $S-N$,
that is, the lower bound $\D'$ is defined using rules \ref{eq:bound-l-odd-sn} or \ref{eq:bound-l-odd-ss},
then the splitting is finished, as $\D$ is contained within the union of $\{1,\}$ and $\{2,\}$.

Now turn to case if \ref{14.5} takes place,
and $\D$ is of type $N-S$ or $N-N$,
so $\D'$ is set with one of the rules \ref{eq:bound-l-odd-ns} or \ref{eq:bound-l-odd-nn}.

%Obtain the condition, equivalent to ``some other condition -- to be clarified''.
Let's see when the subrectangle $\{3, 11\}$ is right-normal.
The following condition should take place:

\begin{equation*}
	\g - \g' > 1,4 (\d - \d'),
\end{equation*}
where
\Tp{28}{22}{68}{65}
which can be written as
\begin{equation}\label{14.6}
	0,345 > q \up{22}{65} \up{28}{68}.
\end{equation}

If the condition \ref{14.6} takes place, then ``here should be a picture''
%\pic{}{}{}
the inequality
\begin{equation*}
	\g - \g' > \d - \d',
\end{equation*}
where
\Tp{25}{3}{66}{63}
which can be written as
\begin{equation}\label{14.7}
	0,322 > q \up{3}{63} \up{25}{66}.
\end{equation}

All in all, if rectangle $\D$ satisfies both conditions \ref{14.6} and \ref{14.7},
then it can be split into $\{1,\}$, $\{2,\}$ and $\{3,\}$.

Now suppose that at least one of \ref{14.6} and \ref{14.7} doesn't take place.
We will show that then we can split $\D$ into $\{1,\}$, $\{2,\}$ and $\{3,1\}$.

First, we need to check goodness of subcovering $\{3,1\}$.
We will prove that it meets the sufficient condition \ref{eq:goodness-condition-3,43}.
We need to show that
\begin{equation}\label{14.8}
	3,43 (\g - \g') > \left| \d - \d' \right|,
\end{equation}
where
\Tp[.]{28}{1}{95}{65}

Inequality \ref{14.8} is equivalent to
\begin{equation}
	0,628 > q \up{1}{65} \up{28}{95}.
\end{equation}

Recalling \ref{14.5}, we have to show that
\begin{equation*}
	0,558 \pu{63}{35} \pu{70}{63}
	< 0,627 \pu{65}{1} \pu{95}{28},
\end{equation*}
and it is true.

Now show that the subrectangles $\{3,1\}$ and $\{2, 1\}$ intersect.

First, let $a_{i_2} = 3$, so $\D_2'$ is set with the formula \ref{eq:bound-l-odd-ns}.

The following inequality should take place:
\begin{equation*}
	\g - \g' < \d - \d',
\end{equation*}
where
\Tp{30}{25}{70}{90}
which is transformed into
\begin{equation}
	0,328 < q \up{25}{70} \up{30}{90}.
\end{equation}

