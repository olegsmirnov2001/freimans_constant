\section{Induction, $i_1 \not\equiv i_2 \pmod 2$}

Let $\D = \{ a_{i_1}, ..., a_{i_2}\}$ be a horizontal rectangle,
satisfying conditions from sections \oldref{sc:boundaries} and \oldref{sc:good}.

This section deals with the case $i_1 \not\equiv i_2 \pmod 2$.

It is easier, because in this case both the lower-left and upper-right corners
are filled with the ``wide'' rectangles $\{1,\}$ and $\{,1\}$ (see figure \oldref{fg:subrectangles-10-01}).
In case $i_1 \equiv i_2 \pmod 2$ the lower-left corner, filled with $\{3,3\}$,
will be a pain in the neck.

Without loss of generality, we may assume that $i_1$ is even and, therefore, $i_2$ is odd.
As the reader will see, the agrument is repeated for the opposite parity,
up to differences in illustrations.

\subsection{Case $\{, 1\}$ is good}

At first, suppose that the subrectangle $\{,1\}$ is good.

It is so, if

\pic[0.5]{induction-odd-2}

\begin{equation*}
	\g - \g' < \d - \d',
\end{equation*}
where
\Tp{66}{63}{70}{90}
which can be rewritten with \ref{eq:aspect-ratio} as follows:
\begin{equation}\label{14.2}
	0,74 < q \up{63}{66} \up{70}{90}.
\end{equation}

Constants $\T{\g}$ and $\T{\d}$ are chosen to ensure that
\begin{equation*}
	\left[ \d, \d' \right] \subset \D_2,
\end{equation*}
regardless of the rules from the section \oldref{sc:boundaries} defining $\D_2'$ and $\D_2''$.

From \ref{14.2} it follows that the original rectangle $\D$ is contained
within the union of the subrectangles $\{1,\}$ and $\{,1\}$.

\pic[0.4][\label{fg:subrectangles-10-01} Subrectangles $\{1,\}$ and $\{,1\}$.]{induction-odd-1}

Now we will try contradiction and suppose that
\begin{equation}\label{14.3}
	0,74 \geqslant q \up{63}{66} \up{70}{90}.
\end{equation}

We will use the inequality \ref{14.3} until the end of the section.


\subsection{Case $\D$ is left-shortened}

Now suppose that the following inequality takes place:
\pic[0.5]{induction-odd-3}
\begin{equation*}
	\g - \g' > \d - \d',
\end{equation*}
where
\Tp{63}{35}{63}{70}
which can be rewritten as
\begin{equation}\label{14.5}
	0,558 > q \up{35}{63} \up{63}{70}.
\end{equation}

If the rectangle $\D$ is of type $S-S$ or $S-N$,
that is, the lower bound $\D'$ is defined using rules \ref{eq:bound-l-odd-sn} or \ref{eq:bound-l-odd-ss},
then the splitting is finished, as $\D$ is contained within the union of $\{1,\}$ and $\{2,\}$.


\subsection{Case $\D$ is left-normal}

Now turn to case if \ref{14.5} takes place,
and $\D$ is of type $N-S$ or $N-N$,
so $\D'$ is set with one of the rules \ref{eq:bound-l-odd-nn} or \ref{eq:bound-l-odd-ns}.

%Obtain the condition, equivalent to ``some other condition -- to be clarified''.
Let's see when the subrectangle $\{3, 11\}$ is right-normal.
The following condition should take place:

\pic[0.5]{induction-odd-4}

\begin{equation*}
	\g - \g' > 1,4 (\d - \d'),
\end{equation*}
where
\Tp{28}{22}{68}{65}
which can be written as
\begin{equation}\label{14.6}
	0,345 > q \up{22}{65} \up{28}{68}.
\end{equation}

\pic[0.5]{pic5}
If the condition \ref{14.6} takes place, then
the inequality
\begin{equation*}
	\g - \g' > \d - \d',
\end{equation*}
where
\Tp{25}{3}{66}{63}
which can be written as
\begin{equation}\label{14.7}
	0,322 > q \up{3}{63} \up{25}{66}.
\end{equation}

All in all, if rectangle $\D$ satisfies both conditions \ref{14.6} and \ref{14.7},
then it can be split into $\{1,\}$, $\{2,\}$ and $\{3,\}$.

Now suppose that at least one of \ref{14.6} and \ref{14.7} doesn't take place.
We will show that then we can split $\D$ into $\{1,\}$, $\{2,\}$ and $\{3,1\}$.

First, we need to check goodness of subcovering $\{3,1\}$.
We will prove that it meets the sufficient condition \ref{eq:goodness-condition-3,43}.
We need to show that
\begin{equation}\label{14.8}
	3,43 (\g - \g') > \left| \d - \d' \right|,
\end{equation}
where
\Tp[.]{28}{1}{95}{65}

Inequality \ref{14.8} is equivalent to
\begin{equation*}
	0,628 > q \up{1}{65} \up{28}{95}.
\end{equation*}

Recalling \ref{14.5}, we have to show that
\begin{equation*}
	0,558 \pu{63}{35} \pu{70}{63}
	< 0,627 \pu{65}{1} \pu{95}{28},
\end{equation*}
and it is true.

Now show that the subrectangles $\{3,1\}$ and $\{2, 1\}$ intersect.

First, let $a_{i_2} = 3$, so $\D_2'$ is set with the formula \ref{eq:bound-l-odd-ns}.

The following inequality should take place:
\begin{equation}\label{14.9}
	\g - \g' < \d - \d',
\end{equation}
where
\Tp{30}{25}{70}{90}
which is transformed into
\begin{equation}\label{14.10}
	0,328 < q \up{25}{70} \up{30}{90}.
\end{equation}

If \ref{14.6} doesn't take place, then we need the following inequality for \ref{14.10}:
\begin{gather*}
	0,345 \pu{65}{22} \pu{68}{28} > 0,328 \pu{70}{25} \pu {90}{30}\\
	\ArrowBetweenLines
	1,05 > \pp {70}{65} \pp{90}{68} \uu{22}{25} \uu{28}{30}.
\end{gather*}

From $a_{i_2} = 3$ it follows that $p' \leqslant \frac{1}{3}$.
Setting $p' = \frac{1}{3}$ and $p = \frac{1}{4}$ in the rhs, increasing it and obtaining the correct inequality.
Thus, it is correct.

Now suppose that \ref{14.6} takes place, while \ref{14.7} doesn't.
Setting $\T{\d} = \T{66}$ (instead of $\T{70}$) in \ref{14.9}, get the following inequality:

\begin{equation}\label{14.11}
	0,288 < q \up{25}{66} \up{30}{90}.
\end{equation}

Inequality \ref{14.11} follows from denial of inequality \ref{14.7}, if
\begin{gather*}
	0,322 \pu{63}{3} \pu{66}{25} > 0,298 \pu{66}{25} \pu{90}{30}\\
	\ArrowBetweenLines
	1,08 > \uu{3}{30} \pp{90}{63}.
\end{gather*}

Setting $p = \frac{1}{4}$ and $p' = \frac{1}{3}$, easily check the last inequality.

\subsection{Case we can consider $\{2, 1\}$}

Now consider the case, when $a_{i_2} \ne 3$ and $\D_1'$ is defined with \ref{eq:bound-l-odd-nn}.

For the subrectangle $\{2, 1\}$ left-normality we need
\begin{equation}\label{14.12}
	1,4 (\g - \g') < \| \d - \d' \|,
\end{equation}
where
\Tp[,]{33}{29}{95}{92}
so \ref{14.12} can be rewritten as
\begin{equation}\label{14.13}
	0,182 < q \up{29}{92} \up{33}{95}.
\end{equation}

If neither \ref{14.6} nor \ref{14.7} takes place, then it's easy to check \ref{14.13}.

In \ref{14.9} we can set $\T{\g'} = \T{94}$ (instead of $\T{90}$), so \ref{14.9} is transformed into
\begin{equation}\label{14.14}
	0,248 < q \up{25}{70} \up{30}{94}.
\end{equation}

If \ref{14.6} doesn't take place, then for \ref{14.14} we will check the following:

\begin{gather*}
	0,345 \pu{65}{22} \pu{68}{28} > 0,248 \pu{70}{25} \pu{94}{30}\\
	\ArrowBetweenLines
	1,39 > \pp{70}{65} \pp{94}{68} \uu{22}{25} \uu{28}{30},
\end{gather*}
which is clear.

If \ref{14.7} doesn't take place, then for \ref{14.14} we will check
\begin{gather*}
	0,322 \pu{63}{3} \pu{66}{25} > 0,248 \pu{70}{25} \pu{94}{30}\\
	\ArrowBetweenLines
	1,29 > \uu{3}{30} \pp{70}{63} \pp{94}{66},
\end{gather*}
which is true.

\subsection{Further plan}

It remains for us to check the case if \ref{14.3} takes place and \ref{14.5} doesn't --
instead, having the negation of \ref{14.5}:
\begin{equation}\label{14.15}
	0,558 \leqslant q \up{35}{63} \up{63}{70}.
\end{equation}

We have proved that we can consider subrectangles $\{1,\}$ and $\{2, 2\}$.
\pic[0.5]{max-2-2}
The picture illustrates the case when
the boundaries $\D_1'$ and $\D_2''$ of segments $\D_1$ and $\D_2$ are defined
with constants $\T{\D_1'} = \T{\D_2''} = \T{30} = \left[0; 213\overline{12}\right]$.
In this case, having, in particular, $a_{i_1} = 1$, $a_{i_1 + 1} = 3$.
We set segment $\D_1$ to be left-shortened to forbid $a_{i_1 - 1} = 3$,
because it can happen that we will face the forbidden combination $(31313)$.

So, in the illustrated case, the only unused subsegment is $\{2, 1\}$.
However, subsegments $\{2, 1\}$ and $\{2, 2\}$ doesn't intersect,
so we will be unable to construct a family of subsegments, covering $\D$.

How to overcome this difficulty?
The idea being developed is to use $a_{i_1 - 1} = 3$ (and $a_{i_2 + 1} = 3$),
but set $a_{i_1 - 2},\, a_{i_1 - 3}$ (and $a_{i_2 + 2},\, a_{i_2 + 3}$),
which produce the thinner subrectangles and doesn't occur in string $(31313)$.

\subsection{List of subrectangles}

Let's turn to formal plan.

If $\D_1'$ and $\D_2''$ were set with $\T{3} = \left[0; 3\overline{12}\right]$,
then we would have been able to take $\{3, 3\}$, which intersects with $\{2, 2\}$.

Instead of $\{3, 3\}$, we will take the following list of subrectangles:
\begin{gather*}
	\{23, 32\}, \{23, 33\} \,(\text{or } \{223, 33\} \text{ and } \{323, 33\}), \{113, 311\},\\
	\{213, 312\}, \{213, 311\}, \{113, 32\}, \{113, 33\} \,(\text{or } \{1113, 33\}), \{213, 33\}.
\end{gather*}
\pic[0.9][
	Illustration of the subrectangles replacing the subrectangle $\{3,3\}$.\\
	I will make a 2d illustration in the future.
]{subrectangles-instead-3-3}

The next subrectangle to consider is $\{3,2\}$, but instead we will consider the following subrectangles:
\begin{equation*}
	\{33, 213\}, \{33, 212\}, \{23, 213\}, \{23, 212\}, \{23, 211\}.
\end{equation*}
