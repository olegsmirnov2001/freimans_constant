\subsection{Subrectangle}
Consider a rectangle $\D$, set by the sequence center
\begin{equation*}
	\left\{ a_{i_1}, a_{i_1 + 1}, ..., a_{i_2} \right\}.
\end{equation*}

We will use a \textbf{shorter notation} for subrectangles, produced by setting integers
$a_i$ for $i < i_1$ or $i > i_2$:
\begin{equation*}
	\left\{ b_\ell...b_1,\; c_1...c_r\right\} \coloneqq
	\left\{ b_\ell...b_1 a_{i_1} ... a_{i_2} c_1...c_r\right\}.
\end{equation*}

For example:
\begin{equation}\tag{ex.1}\label{subrectangle_ex1}
	\left\{213, 3\right\} \coloneqq \left\{ 213 a_{i_1} ... a_{i_2} 3\right\},
\end{equation}
\begin{equation}\tag{ex.2}\label{subrectangle_ex2}
	\left\{2, 0\right\} \coloneqq \left\{ 2 a_{i_1} ... a_{i_2}  \right\}.
\end{equation}

We will also shorter the notation \ref{deltas}: lhs and rhs are $\D_1(312)$ and $\D_2(3)$ for subrectangle \ref{subrectangle_ex1} and $\D_1(2)$ and $a_2$ for \ref{subrectangle_ex2}.
