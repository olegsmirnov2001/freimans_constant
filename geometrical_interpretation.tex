\subsection{Geometrical interpretation}

Consider the mapping 
\begin{gather*}
	\widetilde{h}: \N^\Z \rightarrow \R^2,\\
	\widetilde{h}(\M) = (\g(\M); \d(\M)).
\end{gather*}

In these terms,
the Markov spectrum $M$
is the projection of some subset $\mathcal{S} \subset C_4 \times C_4$ onto the diagonal.

Then \textbf{rectangle} $\D$ is indeed a rectangle $\D_1 \times \D_2$
and \textbf{subrectangles} are its subrectangles.

We will consider a family of rectangles whose projections cover the beginning of Hall's Ray.

Then we will present the algorithm to split rectangle into subrectangles
so that their projections cover the projection of initial rectangle.

When we say that rectangles intersect, we, however, mean that their projections intersect.

The more <<squarish>> the rectangle, the easier the step.

That's why we will bound the aspect ratio of rectangles (see \textbf{good} rectangle).

Formulas to evaluate aspect side lengths and aspect ratio
are given in the section \oldref{section_calculations}.
