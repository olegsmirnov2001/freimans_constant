\subsection{Horizontal rectangle}

\begin{definition}
	Call a rectangle $\D$ \textbf{horizontal}, if
	\begin{equation}
		\label{eq:horizontal_definition}
		|\D_1| \geqslant |\D_2|.
	\end{equation}
\end{definition}

In \ref{eq:horizontal_definition} we allow terms $a_s$ for $s < i_1$ and $s > i_2$
to be integers $\{1, 2, 3\}$,
regardless of the requirement that sequences $\M \in \D$ are centered.

In other words, $\D$ is horizontal, if and only if
\begin{equation*}
	\left|
		\left[ 0; a_{-1}, ..., a_{i_1}, \overline{3, 1} \right] - 
		\left[ 0; a_{-1}, ..., a_{i_1}, \overline{1, 3} \right]
	\right| \geqslant \left|
		\left[ 0; a_{1}, ..., a_{i_2}, \overline{3, 1} \right] -
		\left[ 0; a_{1}, ..., a_{i_2}, \overline{1, 3} \right]
	\right|.
\end{equation*}

Clearly, we can always obtain a horizontal rectangle out of the vertical one,
as we can reindex the sequence in the opposite direction.
