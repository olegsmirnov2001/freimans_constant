\section{Induction, $i_1 \equiv i_2 \pmod 2$}

This section deals with the case $i_1 \equiv i_2 \pmod 2$.
Again, let $\D = \{ a_{i_1}, ..., a_{i_2}\}$ be a horizontal rectangle,
satisfying conditions from sections \oldref{sc:boundaries} and \oldref{sc:good}.

As in section \ref{sc:odd}, we will assume that both $i_1$ and $i_2$ are even;
for the different parity the same logic can be repeated,
with bounds in the inequalities swapped and illustrations reflected.

\subsection{Case $\{3,2\}$ is right-normal}

We start with the case
\begin{equation}\label{15.2}
	\g - \g' > \d - \d',
\end{equation}
where
\Tp{25}{3}{66}{63}

\textit{
	Freiman ceases to give explanatory illustrations to inequalities like \ref{15.2} in this section
	and suggests the reader to make them on their own.
}

Inequality \ref{15.2} expands to
\begin{equation}\label{15.3}
	0,32 > q \up{3}{63} \up{25}{66}.
\end{equation}

The rounding is done so that \ref{15.2} and \ref{15.5} follow from \ref{15.3}.

The constants $\T{25}$ and $\T{3}$ are chosen such that the subrectangle $\{3,2\}$
which will occur in the further splitting is right-normal.
This condition allows us to choose the constant in the lhs of \ref{15.3} as large as possible.

So, the following condition should take place:
\begin{equation}\label{15.4}
	\g - \g' > 1,4(\d - \d'),
\end{equation}
where
\Tp[,]{28}{22}{64}{61}
which expands to
\begin{equation}\label{15.5}
	0,345 > q \up{22}{61} \up{28}{64}.
\end{equation}

Let's show that \ref{15.5} follows from \ref{15.3}:
\begin{gather*}
	0,344 \pu{61}{22} > 0,31 \pu{66}{3}\\
	\ArrowBetweenLines
	1,1 > \uu{22}{3} \pp{66}{61}.
\end{gather*}

Checking it by replacing $p$ and $p'$ with their upper bound 0.8 and increasing the rhs.

In consideration of \ref{15.3},
in case $\D'$ is defined with \ref{eq:bound-l-nnn} and \ref{eq:bound-l-nns},
we can take $\{1,\}$, $\{2,\}$, and $\{3,\}$,
and in case \ref{eq:bound-l-sn} and \ref{eq:bound-l-ss}
we can take $\{1,\}$ and $\{2,\}$.


