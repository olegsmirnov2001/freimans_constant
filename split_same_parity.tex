\section{Induction, $i_1 \equiv i_2 \pmod 2$}

This section deals with the case $i_1 \equiv i_2 \pmod 2$.
Again, let $\D = \{ a_{i_1}, ..., a_{i_2}\}$ be a horizontal rectangle,
satisfying conditions from sections \oldref{sc:boundaries} and \oldref{sc:good}.

As in section \ref{sc:odd}, we will assume that both $i_1$ and $i_2$ are even;
for the different parity the same logic can be repeated,
with bounds in the inequalities swapped and illustrations reflected.

\subsection{Case $\{3,2\}$ is right-normal}

We start with the case
\begin{equation}\label{15.2}
	\g - \g' > \d - \d',
\end{equation}
where
\Tp{25}{3}{66}{63}

\textit{
	Freiman ceases to give explanatory illustrations to inequalities like \ref{15.2} in this section
	and suggests the reader to make them on their own.
}

Inequality \ref{15.2} expands to
\begin{equation}\label{15.3}
	0,31 > q \up{3}{63} \up{25}{66}.
\end{equation}

The rounding is done to a higher precision
so that \ref{15.2} and \ref{15.5} both follow from \ref{15.3}.

The constants $\T{25}$ and $\T{3}$ are chosen such that the subrectangle $\{3,2\}$
which will occur in the further splitting is right-normal.
This condition allows us to choose the largest possible constant in the lhs of \ref{15.3}.

So, the following condition should take place:
\begin{equation}\label{15.4}
	\g - \g' > 1,4(\d - \d'),
\end{equation}
where
\Tp[,]{28}{22}{64}{61}
which expands to
\begin{equation}\label{15.5}
	0,345 > q \up{22}{61} \up{28}{64}.
\end{equation}

Let's show that \ref{15.5} follows from \ref{15.3}:
\begin{gather}\label{15.6}
	0,344 \pu{61}{22} > 0,31 \pu{66}{3}\\
	\ArrowBetweenLines
	1,1 > \uu{22}{3} \pp{66}{61}.
\end{gather}

Checking it by replacing $p$ and $p'$ with their upper bound 0.8 and increasing the rhs.

In consideration of \ref{15.3},
in case $\D'$ is defined with \ref{eq:bound-l-nnn} and \ref{eq:bound-l-nns},
we can take $\{1,\}$, $\{2,\}$, and $\{3,\}$,
and in case \ref{eq:bound-l-sn} and \ref{eq:bound-l-ss}
subrectangles $\{1,\}$ and $\{2,\}$ are enough.

In the rest section we will assume that \ref{15.3} doesn't take place, that is,
\begin{equation}\label{15.7}
	0,31 \leqslant q \up{3}{63} \up{25}{66}.
\end{equation}

\subsection{Name}

Consider condition
\begin{equation}\label{15.8}
	\g - \g' > \d - \d',
\end{equation}
where
\Tp[,]{63}{36}{66}{63}
which expands to
\begin{equation}\label{15.9}
	0,634 > q \up{36}{63} \up{63}{66}.
\end{equation}

To justify the choice $\T{\g} = \T{\g'} = \T{63}$, we need to check
\begin{equation}\label{15.10}
	\g - \g' > 1,4 (\d - \d'),
\end{equation}
where
\Tp[,]{64}{61}{64}{61}
which expands to
\begin{equation}\label{15.11}
	0,714 > q\up{61}{61}\up{64}{64}.
\end{equation}
The last inequality follows from \ref{15.9}, if
\begin{equation}
	\pp{63}{61} \pp{66}{64} \uu{61}{36} \uu{64}{63} < 1,125,
\end{equation}
which is easily checked by substituting $p = p' = 0.8$, as in \ref{15.6}.

If \ref{15.8} takes place, then take subrectangles $\{1,\}$ and $\{2,\}$.
Thus, in consideration of \ref{15.9}, cases \ref{eq:bound-l-sn} and \ref{eq:bound-l-ss} are over.

Now turn to cases \ref{eq:bound-l-nnn} and \ref{eq:bound-l-nns} (IIa).
We will show that
\begin{equation}\label{15.12}
	\g - \g' < \d - \d',
\end{equation}
where
\Tp[,]{30}{25}{59}{3}
which can be rewritten as
\begin{equation}\label{15.13}
	0,3375 < q \up{25}{3} \up{30}{59}.
\end{equation}

The last inequality follows from \ref{15.7}, if
\begin{equation*}
	1,09 < \pp{63}{3} \pp{66}{59} \uu{30}{3},
\end{equation*}
which is checked by substituting $p = p' = 0,25$.

The inequality \ref{15.12} shows that, in consideration of \ref{15.9} and in case \ref{eq:bound-l-nnn},
the subrectangle $\{3,2\}$ intersects with $\{2,\}$, even if $\{3,2\}$ is right-shortened.

If $\D$ is left-shortened, then the splitting is over.
If it is left-normal, then subrectangles $J_k = \{\overset{k}{3}, \overset{k}{3}\}$ are added.

Check that, for example, $J_1 = \{3, 3\}$ is good.

We need
\begin{equation}\label{15.14}
	\g - \g' < 3,8 (\d - \d'),
\end{equation}
where
\Tp[,]{28}{1}{28}{1}
which expands to
\begin{equation}
	0,264 < q \up{1}{1} \up{28}{28}
\end{equation}
and follows from \ref{15.7}, if
\begin{equation*}
	1,17 > \uu{3}{1} \uu{25}{28} \pp{1}{63} \pp{28}{66},
\end{equation*}
which is obvious.

\textit{
	Here Freiman links the $\sqrt{21}$ proof from \S9.
	I will translate it soon.
}

It remains to deal with \ref{eq:bound-l-nns} (IIb).

Suppose that the following inequality takes place:
\begin{equation}\label{15.15}
	\g - \g' < \d - \d'
\end{equation}
with
\Tp[.]{30}{25}{63}{36}

Inequality \ref{15.15} is rewritten as follows.
Instead of 0,703 in the lhs take constant 0,73 to assure $\{3, 2\}$ normality:
\begin{equation}\label{15.16}
	0,73 < q \up{25}{36} \up{30}{63}.
\end{equation}

Subrectangle $\{3, 2\}$ is normal, if
\begin{equation*}\label{15.17}
	1,4 (\g - \g') < \d - \d',
\end{equation*}
where
\Tp[,]{28}{22}{64}{61}
which expands to
\begin{equation*}
	0,6773 < q \up{22}{61} \up{28}{64}
\end{equation*}
and follows from \ref{15.16}, if
\begin{equation*}
	1,076 > \pp{61}{36} \pp{64}{63} \uu{30}{22},
\end{equation*}
which is checked as \ref{15.6}.

In addition to subrectangles $\{1,\}$ and $\{2,\}$ we can now take $\{3, 2\}$,
and case \ref{eq:bound-l-nns} is done, if \ref{15.16}, \ref{15.9}, and \ref{15.7} all take place.

Note that \ref{15.16} does not depend on \ref{15.9}.
We will need it below.
If \ref{15.16} doesn't take place, that is,
\begin{equation}\label{15.18}
	0,73 \geqslant q \up{25}{36} \up{30}{63},
\end{equation}
then in case \ref{eq:bound-l-nns} the subrectangles $\{2,\}$ and $\{3, 2\}$ may not intersect when
\begin{equation*}
	\D_1'(2) + \D_2'(2) > \D_1''(3) + \D_2''(2).
\end{equation*}

The system of subrectangles covering the interval $\left(\D_1''(3) + \D_2''(2); \D_1'(2) + \D_2'(2)\right)$
will be found further.

Suppose that \ref{15.9} doesn't take place, that is,
\begin{equation}\label{15.19}
	0,634 \leqslant q \up{36}{63} \up{63}{66}.
\end{equation}

At first, consider subrectangles $\{1,\}$ and $\{2, 1\}$.
Rectangle $\D$ is good, so they intersect.

Showing that $\{2, 1\}$ is good.
We need the following:
\begin{equation}
	3,8 (\g - \g') > \d - \d',
\end{equation}
where
\Tp[,]{64}{29}{95}{65}
which expands to
\begin{equation*}
	1,36 > q \up{29}{65} \up{64}{95}
\end{equation*}
and is easily checked.

Now consider subrectangle $\{3, 1\}$, if it is good.
The following is required:
\begin{equation*}
	3,8 (\g - \g') > \d - \d',
\end{equation*}
where
\Tp[,]{28}{1}{95}{65}
which expands to
\begin{equation}\label{15.20}
	0,696 > q \up{1}{65} \up{28}{95}.
\end{equation}

Let the negation of \ref{15.20}, the inequality \ref{15.21}, take place:
\begin{equation}\label{15.21}
	0,696 \leqslant q \up{1}{65} \up{28}{95}.
\end{equation}

Then considering the subrectangle $\{3, 11\}$ and showing that it intersects with $\{2, 1\}$.
The following inequality is needed:
\begin{equation}\label{15.22}
	\g - \g' < \d - \d',
\end{equation}

% subsections required
