\subsection{Shortened rectangle}
\label{sbsc:shortened}

Consider a horizontal rectangle $\D$.


\subsubsection{Case $i_1 \equiv i_2 \equiv 0 \pmod 2$}
\label{sbsbsc:shortened_even}

If both $i_1$ and $i_2$ are even, then left- and right-shortened rectangles are defined as follows.

\begin{definition}
	Rectangle $\D$ is called left-shortened,
	if for subrectangle $\{3, 3\}$ the following condition takes place:
	\begin{equation}
		\label{eq:left-shortened}
		\left| \D_1(3) \right| \leqslant 1.4 \cdot \left| \D_2(3) \right|.
	\end{equation}
\end{definition}

Here, as in \ref{eq:horizontal_definition},
we allow terms $a_s$ for $s < i_1$ and $s > i_2$ to be integers $\{1, 2, 3\}$,
so \ref{eq:left-shortened} can be rewritten as
%
\begin{equation*}
	\hspace*{-1pt}
	\left|
		\left[ 0; a_{-1}, ..., a_{i_1}, 3, \overline{3, 1} \right] -
		\left[ 0; a_{-1}, ..., a_{i_1}, \overline{3, 1} \right]
	\right| \leqslant 1.4 \,\cdot\, \left|
		\left[ 0; a_{1}, ..., a_{i_2}, 3, \overline{3, 1} \right] -
		\left[ 0; a_{1}, ..., a_{i_2}, \overline{3, 1} \right]
	\right|.
	\hspace*{-1pt}
\end{equation*}

For further convenience, we also introduce the opposite to \ref{eq:left-shortened} condition:

\begin{definition}
	Rectangle $\D$ is called left-normal, if
	\begin{equation}
		\label{eq:left-normal}
		\left| \D_1(3) \right| > 1.4 \cdot \left| \D_2(3) \right|.
	\end{equation}
\end{definition}

Now we introduce the notion of right-shortened rectangle:

\begin{definition}
	Rectangle $\D$ is called right-shortened,
	if for subrectangle $\{31, 13\}$ the following condition takes place:
	\begin{equation}
		\label{eq:right-shortened}
		\left| \D_1(13) \right| \leqslant 1.4 \cdot \left| \D_2(13) \right|.
	\end{equation}
\end{definition}

As in \oldref{sbsbsc:shortened_even},
we allow terms $a_s$ for $s < i_1$ and $s > i_2$ to be integers $\{1, 2, 3\}$,
so \ref{eq:right-shortened} can be rewritten as
%
\begin{equation*}
	\hspace*{-12pt}
	\left|
		\left[ 0; a_{-1}, ..., a_{i_1}, 1, 3, \overline{3, 1} \right] -
		\left[ 0; a_{-1}, ..., a_{i_1}, \overline{1, 3} \right]
	\right| \leqslant 1.4 \,\cdot\, \left|
		\left[ 0; a_{1}, ..., a_{i_2}, 1, 3, \overline{3, 1} \right] -
		\left[ 0; a_{1}, ..., a_{i_2}, \overline{1, 3} \right]
	\right|.
	\hspace*{-12pt}
\end{equation*}

We also introduce the opposite condition:

\begin{definition}
	Rectangle $\D$ is called right-normal, if
	\begin{equation}
		\label{eq:right-normal}
		\left| \D_1(13) \right| > 1.4 \cdot \left| \D_2(13) \right|.
	\end{equation}
\end{definition}


\subsubsection{Other cases}

If $i_1 \equiv i_2 \pmod 2$ and $i_1$ is odd,
then the conditions for left- and right-shortened rectangles are swapped.

In case $i_1 \not\equiv i_2 \pmod 2$,
the notions of left- and right-shortened rectangles
are inferred from subrectangles $\{1,\}$, $\{2,\}$, and $\{3,\}$.

We will refer to the relevant conditions from \oldref{sbsc:boundaries_formal}
