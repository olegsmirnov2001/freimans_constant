\section{Corollaries}

\subsection{Segment in $C(3) \backslash (31313) \oplus C(3) \backslash (31313)$}

The segments in $C(3) \backslash (31313) \oplus C(3) \backslash (31313)$
are likely to indicate the presence of segments inside the Markov spectrum, prior to Freiman's constant.
Thus such corollary can be useful.

\subsubsection{Freiman's segment in Hall's Ray approach}

Freiman covers the segment $[\mu; \mu + \varepsilon]$
with the projections of rectangles with the center
\begin{equation}
	\label{eq:center-44}
	(... \underset{0}{4} 4 ...),
\end{equation}
and performs the splitting, adding only terms $(1, 2, 3)$
and forbidding the substring $(31313)$.

Clearly, under the assumption that the Freiman's splitting process is correct,
the following segment is contained in the almost desired sum of the Cantor's sets:
\begin{equation*}
	[\mu - 4; \mu - 4 + \varepsilon] \subset C(3) \backslash (31313) \oplus \dfrac{1}{4 + C(3) \backslash (31313)}.
\end{equation*}

It remains to evaluate $\varepsilon$.

As presented by Figure \oldref{initial-set-table},
all the rectangles $5, 6, 7, (8, ...), (9, ...), (10, ...)$ are of type \ref{eq:center-44}.

The union of their projections cover the segment up to $4.52832$.
Thus, for
\begin{equation*}
	\varepsilon = 4.52832 - \mu \approx 0.0005
\end{equation*}
having:
\begin{equation}
	[\mu - 4; \mu - 4 + \varepsilon] \approx [4.52782; 4.52832] \subset
	C(3) \backslash (31313) \oplus \dfrac{1}{4 + C(3) \backslash (31313)}.
\end{equation}

This is the tiny interval, however, it was used by Freiman
while proving the Freiman's constant.


\subsubsection{Freiman's splitting process approach}

We will construct a rectangle,
which is good, according to the section \oldref{sc:good} rules.

Therefore its projection will preserve throughout the Freiman's splitting process.

I will consider the rectangle with the empty center:
\begin{equation*}
	\{\underset{0}{0}\}
\end{equation*}

Section \oldref{sc:boundaries} provides the following boundaries for this rectangle:
\begin{align*}
	\D' & = \l(\overline{21}3\underset{0}{0}213\overline{12}) = 2\left(2 - \sqrt{3}\right) \approx 0.62654,\\
	\D'' & = \l(\overline{21}31\underset{0}{0}1213\overline{12}) = 1 + \tfrac{1}{\sqrt{3}} \approx 1.52473.
\end{align*}

Clearly, the rectangle $\D$ is good, since it is almost a square and its aspect ratio is almost $1:1$.

Therefore, the segment $[\D'; \D'']$ is contained within the sum of the desired Cantor sets:
\begin{equation}
	[\D'; \D''] \subset C(3) \backslash (31313) \oplus C(3) \backslash (31313).
\end{equation}

The convex hull of the sum is the following segment:
\begin{equation*}
	C(3) \backslash (31313) \oplus C(3) \backslash (31313) \subset
	\left[ 2 \cdot \left[0; \overline{3, 1, 3, 1, 2, 1}\right]; \left[0; \overline{1, 3, 1, 3, 1, 2}\right] \right] \approx
	\left[0.52786; 1.58236\right]
\end{equation*}

The length of the projection of $\D$ is
\begin{equation}
	\label{eq:length-segment}
	\left|\D\right| \approx 0.89819,
\end{equation}
which is $85\%$ of the length of the convex hall of the sum set:
\begin{gather*}
	\left|\overline{C(3) \backslash (31313) \oplus C(3) \backslash (31313)}\right| \approx 1.0545.
\end{gather*}

However, \ref{eq:length-segment} does not give any statement about the Markov Spectrum directly,
because, for example, for rectangle with the subcenter
\begin{equation*}
	\{\underset{0}{4}1\underset{2}{3}12\}
\end{equation*}
the height function at the position 2 may be too large and should be considered,
when talking about the spectrum, not the Cantor set.
