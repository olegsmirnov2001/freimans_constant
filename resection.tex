\subsection{Resection}
\label{sbsc:resection}

\begin{definition}
	Call \textbf{resection} of a segment $A = [a; b]$
	a process of removing subsegment $A_{12} = [a_1; b_1]$,
	leaving two segments $A_1 \sqcup A_2 = [a; a_1] \sqcup [b_1; b]$.
\end{definition}

%We will introduce an infinite sequence of resections
%(at first segment $A$ is resected, then resections are performed at $A_1$ and $A_2$, and so on)
%which will produce Cantor set $\mathcal{L}(A)$.

\begin{definition}
	Call subsegment $A_{12} \subset A$ \textbf{normal}, if it is thicker than the two remaining subsegments:
	\begin{equation}
		\label{normal_resection}
		|A_{12}| \leqslant min\;\{|A_1|,\; |A_2|\}
	\end{equation}
\end{definition}

We call a resection \textbf{normal} if the resected subsegment is normal.

\begin{proposition}
	For any normal resection, having
	\begin{equation}
		\label{sum_is_present}
		A + A = (A_1 \sqcup A_2) + (A_1 \sqcup A_2).
	\end{equation}
\end{proposition}

%Clearly, if segment $A$ is transformed into Cantor set $\mathcal{L}(A)$ with only normal resections, then
%\begin{equation*}
%	A + A = \mathcal{L}(A) + \mathcal{L}(A).
%\end{equation*}

%\textit{We will \textbf{only} perform normal resections during the proof.}
