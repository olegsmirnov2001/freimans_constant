\begin{frame}
\frametitle{Markov value}

\begin{definition}
	Let the Markov value $M(\alpha)$ of an irrational $\alpha$ be
	the smallest constant $c$ such that inequality (\ref{eq:markov-inequality}) has infinitely many solutions.
\end{definition}

%By Hurwitz theorem, $M(\alpha) \geqslant \sqrt{5}$ for all $\alpha$ and $M(\varphi) = \sqrt{5}$.

Markov values allow us to introduce Lagrange spectrum.

\begin{definition}
	Lagrange spectrum $L$ is the set of the Markov values over all irrationals:
	\begin{equation*}
		L \coloneqq \{M(\alpha) \mid \alpha \in \R \backslash \Q\}.
	\end{equation*}
\end{definition}

\end{frame}
